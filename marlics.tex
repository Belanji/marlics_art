%% This template can be used to write a paper for
%% Computer Physics Communications using LaTeX.
%% For authors who want to write a computer program description,
%% an example Program Summary is included that only has to be
%% completed and which will give the correct layout in the
%% preprint and the journal.
%% The `elsarticle' style is used and more information on this style
%% can be found at 
%% http://www.elsevier.com/wps/find/authorsview.authors/elsarticle.
%%
%%
%%\documentclass[preprint,12pt]{elsarticle}

%% Use the option review to obtain double line spacing
%% \documentclass[preprint,review,12pt]{elsarticle}

%% Use the options 1p,twocolumn; 3p; 3p,twocolumn; 5p; or 5p,twocolumn
%% for a journal layout:
%% \documentclass[final,1p,times]{elsarticle}
%% \documentclass[final,1p,times,twocolumn]{elsarticle}
%% \documentclass[final,3p,times]{elsarticle}
%% \documentclass[final,3p,times,twocolumn]{elsarticle}
%% \documentclass[final,5p,times]{elsarticle}
 \documentclass[final,5p,times,twocolumn]{elsarticle}


\usepackage{amsmath, amssymb,amsbsy,amsfonts}
\usepackage{color}



\usepackage{longtable}
\usepackage{listings}
\usepackage[caption=false]{subfig}

%% if you use PostScript figures in your article
%% use the graphics package for simple commands
%% \usepackage{graphics}
%% or use the graphicx package for more complicated commands
\usepackage{graphicx}
%% or use the epsfig package if you prefer to use the old commands
%% \usepackage{epsfig}


%% The lineno packages adds line numbers. Start line numbering with
%% \begin{linenumbers}, end it with \end{linenumbers}. Or switch it on
%% for the whole article with \linenumbers after \end{frontmatter}.
 \usepackage{lineno}

%% natbib.sty is loaded by default. However, natbib options can be
%% provided with \biboptions{...} command. Following options are
%% valid:

%%   round  -  round parentheses are used (default)
%%   square -  square brackets are used   [option]
%%   curly  -  curly braces are used      {option}
%%   angle  -  angle brackets are used    <option>
%%   semicolon  -  multiple citations separated by semi-colon
%%   colon  - same as semicolon, an earlier confusion
%%   comma  -  separated by comma
%%   numbers-  selects numerical citations
%%   super  -  numerical citations as superscripts
%%   sort   -  sorts multiple citations according to order in ref. list
%%   sort&compress   -  like sort, but also compresses numerical citations
%%   compress - compresses without sorting
%%
%% \biboptions{comma,round}

% \biboptions{}

%% This list environment is used for the references in the
%% Program Summary
%%
\newcounter{bla}
\newenvironment{refnummer}{%
\list{[\arabic{bla}]}%
{\usecounter{bla}%
 \setlength{\itemindent}{0pt}%
 \setlength{\topsep}{0pt}%
 \setlength{\itemsep}{0pt}%
 \setlength{\labelsep}{2pt}%
 \setlength{\listparindent}{0pt}%
 \settowidth{\labelwidth}{[9]}%
 \setlength{\leftmargin}{\labelwidth}%
 \addtolength{\leftmargin}{\labelsep}%
 \setlength{\rightmargin}{0pt}}}
 {\endlist}

\journal{Computer Physics Communications}

\newcommand{\consoleline}[2][0.5cm]
{\vspace{#1}
\textit{{#2}}
\vspace{#1}
}


\begin{document}

\begin{frontmatter}

%% Title, authors and addresses

%% use the tnoteref command within \title for footnotes;
%% use the tnotetext command for the associated footnote;
%% use the fnref command within \author or \address for footnotes;
%% use the fntext command for the associated footnote;
%% use the corref command within \author for corresponding author footnotes;
%% use the cortext command for the associated footnote;
%% use the ead command for the email address,
%% and the form \ead[url] for the home page:
%%
%% \title{Title\tnoteref{label1}}
%% \tnotetext[label1]{}
%% \author{Name\corref{cor1}\fnref{label2}}
%% \ead{email address}
%% \ead[url]{home page}
%% \fntext[label2]{}
%% \cortext[cor1]{}
%% \address{Address\fnref{label3}}
%% \fntext[label3]{}

\title{Marlics: A finite difference liquid crystal simulation package}

%% use optional labels to link authors explicitly to addresses:
%% \author[label1,label2]{<author name>}
%% \address[label1]{<address>}
%% \address[label2]{<address>}

\author[a]{R. F. de Souza \corref{R. F. de Souza/Rafael S. Zola}}
\author[a]{E. K. Omori}
\author[a,b]{R. S. Zola}

\cortext[author] {Corresponding author.\\\textit{E-mail address:} EuOuZola@somewhere.edu}
\address[a]{Departamento de Física, Universidade Estadual de Maringá, Avenida Colombo 5790,
87020 - 900 Maringá – PR, Brazil}
\address[b]{Universidade Tecnologica Federal do Paraná, Rua Marcilio Dias 635, 86812-460 Apucarana, Paraná,
Brazil}

\begin{abstract}
  In this paper we present Marlics to the world. Marlics is a software
  written in C++ to solve the Berris-Edwards equation of
  nematodynamics without flow for both nematic and cholesteric liquid
  crystals. The program takes as input an descriptive file giving the
  simulations parameters and initial conditions generating a series of
  different snapshots. The code is organized in class modules which
  can be modified by the user base to attend their further needs.
  \textcolor{red}{Review the abstract after the paper is finished}
\end{abstract}

\begin{keyword}
%% keywords here, in the form: keyword \sep keyword
Liquid crystals \sep Landau-de Gennes \sep finite differences.

\end{keyword}

\end{frontmatter}

%%
%% Start line numbering here if you want
%%
 \linenumbers

% Computer program descriptions should contain the following
% PROGRAM SUMMARY.

{\bf PROGRAM SUMMARY}
  %Delete as appropriate.

\begin{small}
\noindent
{\em Program Title: MarLicS}\\

{\em Licensing provisions: GNU General Public License v3.0 (GPL) }\\

{\em Programming language:C++}\\

{\em Supplementary material: Complete instructions about program usage can be found in the user-guide. }\\
  % Fill in if necessary, otherwise leave out.

{\em Nature of problem: Marlics was developed to simulate liquid crystal devices via solution of the Berris-Edwards system of differential equations equations without flow.}\\
% Describe the nature of the problem here. \\

{\em Solution method:  The system of equations is solved using finite differences in both time and space. The time integration is performed using an explicit integrator with or without variable time-step.}\\
% Describe the method solution here.

{\em External routines: The code needs the GSL (Gnu scientific library), an implementation of the CBLas library and an implementation of the OpenMp library(optional).}\\


{\em Running time: From minutes to hours depending on the problem size.}

{\em Computer: Single or multi-core processor with shared memory.}\\

{\em RAM: From hundreds of megabytes to gigabytes depending on the problem size.}\\


{\em Restrictions: The code is parallelized using OpenMp, consequently it can only be run in parallel with shared memory  processors.}\\

{\em Additional comments: The source code comes with a Mathematica
  notebook with can be used to aid the user to implement additional
  interactions, boundary conditions or other situations situation not
  covered by the current software.}

\end{small}


%% main text
\section{Introduction}
\label{Introduction}

Liquid crystals are excellent materials to perform these functions,
since it can selectively reflect or transmit the incoming light
depending on its state, which can be controlled by an external
perturbation. In displays, one of the most common application, it used
of electric fields the control the liquid crystals state \cite{}. The
difference among the many types of displays is due to its boundary
conditions, with strongly interferes in the device operation \cite{}.

The design of new devices requires a great amount of experimentation
and empirical knowledge, which would require a unpreactical number of
experiments to be performed. As an alternative, the researcher can
turn to modeling softwares to aid in the discovery process. Before
prototyping a device it can be simulated in a package in many forms
before being prototyped, saving time and resources.

There are available some commercial softwares available for simulation
of liquid crystals devices, for instance \cite{} and \cite{}. These
softwares provides out of the box capabilities for simulation and
visualization, however it lacks the possibility for extension by the
user, also, if the project is discontinued, there is no way to
implement new features. Recently, an open source software was released
to search for the minimum energy of liquid crystal\cite{Beller_arxiv},
however the code is focused in the equilibrium state of the sample.

In this paper we present \textit{marlics}, it is a open source code
designed to simulate the dynamics of liquid crystal order
parameters. The code The code is written in C++ using an independent
system of classes, which provide the building blocks for the most
common system, but also, it is easy to extend for cases not covered by
the actual program.

\section{Theoretical background}

The scalar order parameters $\lbrace S, P \rbrace$ and the director and co-director $\lbrace \vec{n}, \vec{l} \rbrace$ can be
combined in unique order parameter $Q_{ij}$ , which is a second rank tensor whose elements are given by:
\begin{equation}\label{eq:tensorial_parameter}
  Q_{ij}=\dfrac{1}{2} S ( 3 n_{i} n_{j}- 1) + P (l_i l_j - m_i m_j)
\end{equation}
where $i=1,2,3$ and $j=1,2,3$.  This order parameter is symmetric and traceless, therefore only 5 independent elements, for example $\lbrace Q_{11}, Q_{12},Q_{13}, Q_{22}, Q_{23} \rbrace $ are necessary to fully determine it.

Deviations from the equilibrium value of the scalar order paramter, or
spatial variations of the director has associate and energy density
given by Landau- de Gennes energy density ($f_{LDG}(\mathbf{Q})$:
\begin{align}\label{eq:Landau_deGennes} \nonumber
  f_{LDG}(\mathbf{Q})&=\dfrac{a}{2}(T-T^*) \text{Tr}(\mathbf{Q}^2) +
                       \dfrac{B}{3} \text{Tr}(\mathbf{Q}^3)
                       +  \dfrac{C}{4} \text{Tr}(\mathbf{Q}^2)^2 \\ \nonumber
                     &+ \dfrac{1}{2} L_1 \left( \partial_i Q_{jk} \right)
                       \left( \partial_i Q_{jk} \right) + \dfrac{1}{2} L_2
                       \left( \partial_i Q_{ji} \right) \left( \partial_k Q_{jk} \right) \\
                     &+\dfrac{1}{2} L_3 Q_{ij}\left( \partial_i Q_{kl}\right) \left(\partial_j Q_{kl}\right)  + \dfrac{4 \pi}{P_0} L_q \epsilon_{ijk} Q_{ij}\left( \partial_j Q_{ik} \right)                     
\end{align}
%
where $Q_{ij,k}= \partial Q_{ij}/\partial x_k$ , $\epsilon_{ijk}$ is
the Levi-Civita tensor, $\lbrace L_1 ,L_2,L_3, L_q, L_s \rbrace$ are
the elastic constants, $\lbrace a,B,C \rbrace$ are thermodynamic
constants related to the nematic isotropic transition and $T$ and
$T^*$ are the system temperature and the virtual nematic-isotropic
phase transition temperature, respectively. Here we used Einstein
summation convention in repeated indexes.

The dielectric energy density is given by:
\begin{equation}
 f_e(\mathbf{Q})= -\dfrac{1}{3} \epsilon_0 \Delta \epsilon^m E_i E_j Q_{ij}+ \dfrac{2} \epsilon_0 \mathbf{E} \cdot \mathbf{E},
\end{equation}
where $e_0$ is the vacuum dieletric constant,
$\Delta \epsilon=\epsilon_{\parallel} - \epsilon_{\perp}$ is the
dielectric anisotropy which measures the diference between the
dieletric constant parallel($\epsilon_{\parallel}$) and perpendicular
($\epsilon_{\perp}$) to the liquid crystal director.  The volume energy
density ($f_v$) will be given by the sum of all energy terms being considered:
\begin{equation}
  f_v(\mathbf{Q})=f_{ldg}(\mathbf{Q})+f_e(\mathbf{Q})
\end{equation}

The liquid crystal also interacts with the confining surfaces, which
can induce an order parameter at the surface and a proffered direction
for the direction, which is called surface easy axis $n_0$. One of the
simplest form of the surface energy density between a liquid crystal
and a surface is given by the Rapini-Papoular (also called Nobili-Durant) which is given by:
\begin{equation}
  f_{rp}(\mathbf(Q))=\dfrac{1}{2} W_1 (Q_{ij}-Q^0_{ij}) (Q_{ij}-Q^0_{ij})
\end{equation}
where $Q_{ij}^0$ is the surface induced order parameter, which can be
given in its tensorial form, or constructed using the induced scalar
order parameters $\lbrace S^0, P^0 \rbrace$ and the induced easy axis
$\lbrace \vec{n}, \vec{l} \rbrace$ using expression
\ref{eq:tensorial_parameter}.

When the liquid crystals boundary is a liquid or gas, instead of
inducing a preferred direction the surface may induces a a preferred
plane of orientation perpendicular to the surface. Any variation of
the director inside this plane gives the same energy. This type of
surface energy is described by the Fournier-Galatola anchoring energy
given by \cite{Sec2012}:
\begin{equation} \label{eq:penalizacao}
f_{FG}(\mathbf{Q})=W\left( \tilde{Q_{ij}}(\bold{Q}) - \tilde{Q}_{ij}^{\perp}(\bold{Q}) \right)\left( \tilde{Q_{ij}}(\bold{Q}) - \tilde{Q}_{ij}^{\perp}(\bold{Q}) \right)
\end{equation}
where $W$ is the anchoring strength constant,
$\tilde{Q_{ij}}(\bold{Q})=Q_{ij}+\dfrac{S_0}{3} \delta_{ij}$ and
$\tilde{Q_{ij}}^{\perp}(\bold{Q})=(\delta_{ik}-\nu_i \nu_k)Q_{kl}
(\delta_{lj}-\nu_l \nu_j)$. Here $\vec{\nu}=\lbrace\nu_1,\nu_2,\nu_2 \rbrace$ is the normal surface vector.

The time evolution of the order parameter is given by the
Beris-Edwards set of equations. If we neglect the liquid crystal flow,
the time evolution of $Q_{ij}$ in the bulk will be given by:
\begin{equation} \label{eq:dissipacao_Q}
\mu \dfrac{\partial Q_{ij}}{ \partial t} = -\left( \dfrac{\partial f_{V}(\mathbf{Q})}{\partial Q_{ij}} - \dfrac{\partial}{\partial x_k } \dfrac{\partial f_V(\mathbf{Q})}{\partial Q_{ij,k}}  \right),
\end{equation}
%
where $\mu$ is the bulk viscosity and $ \partial Q_{ij} \partial Q{kl} = (\delta_{ik} \delta_{jl}+\delta_{jk} \delta_{kl}-2 \delta_{ij} \delta_{kl}/3)$. Meanwhile the dynamics in the bulk will be given by
\begin{align}
\mu_s \dfrac{\partial Q_{ij}}{\partial t}=-\left(\nu_k\dfrac{\partial f_{LDG}(\mathbf{Q})}{\partial Q_{ij,k}} -\dfrac{\partial f_{pen}(\mathbf{Q})}{\partial Q_{ij}}\right),
\end{align}
here $\mu_s$ is the surface viscosity.

We solve solve the system of equations using the method of lines
\cite{}. The spatial discretization ids performed by means of finite
differences. In the bulk we have taken central differences for all
first derivatives and symmetric second order finite differences for
the second order derivative. This frameworks allow the implementation
of various integrators with easy, we have provided 3 explicit
integrator with the program: forward Euler, explicit second order
Runge-Kutta and the Dormand-Prince 5(4).

\section{Software Usage:} \label{sec:software_usage}

Here we present the basic information necessary to install and use the
software. The complete information about software usage can be found
in the supplementary material ``Userguide''.

\subsection{Installation:}

The installation of Marlics in Unix systems is very
straightforward. The source code comes with a \textit{makefile} to
help user compile it in its computer. Actually the makefile provides
an automatic installation for two compilers (being one of them free).
The user will need just to assure he/she has the \textit{make}
software and the necessary external libraries and their respective
developer files installed. These libraries are: GSL (Gnu Scientific
library), OpenMp(optional, but highly recomended) and a CBLAS
implementation (you can use the GSL implementation for example). If
everything is present, the user just need to open a terminal in the
program folder and type:
\begin{lstlisting}
  make
\end{lstlisting}
to compile the program. Once the compilation is done, the user will
find a executable named \textit{marlics} in the instalation
folder. For ease of use, this executable can be added to one of the
system search paths for binaries files, or the user can add the
installation folder to the list of search-able paths. Its is also
possible to run the simulations in the same path that the software is
installed, but we strong recommend against, since it can be become
very confuse.

\subsection{Simulation set up and Execution:}

To execute marlics you must call the program passing an input file
which sets the simulation parameter as follows:
\begin{lstlisting}
marlics intput_file
\end{lstlisting}
being \textit{input\_file} the file containing the simulation
parameters.


We already provide input files for some situations that the user can
use to test the program, or as a base to their own simulations. All
the parameters necessary to set up the simulation must be passed to
the program via the input file. An entry in the input file is set by passing the parameter name
followed by the parameter required values:
\begin{lstlisting}
parameter value
\end{lstlisting}

Comments can be placed in the input file starting a line with ``\#'',
everything in this line will be ignored by marlics. Also, everything
following the required parameters will also be ignored by marlics. We
found it quite useful to let the parameters units after its values.

For more details see the
supplementary file ``Userguide''.
For reference, we also include an
example of input file in the appendix \ref{apx:input_file}.

\subsection{Output files:}

The software produces two outputs: an log of the program execution,
and a series of files containing the spatial distribution of the order
parameters. The log has the function of informing the user about the
parameters read by the program, so it can be used as reference in the
future, and informing the current state of the simulation. The log is
printed in the standard output, which can aid the preparation of the
input file, but we strong recommend redirecting it to a separate file
for reference in future.


The main output of the software are the files containing the spatial
distribution of the order parameter: the main LC director director
$\vec{n}$, the co-director $\vec{l}$, the uniaxial order parameter $S$
and biaxial order parameter $P$. We decided to output the order
parameters in this form instead of the elements of $Q_{ij}$, since the
former are more ready to use and interpret than the later.  We also
preferred to refer to the position in space using the lattice numbers
instead of the space position in the Cartesian frame. The actual
position can be easily reprieved multiplying the column by its
referred grid spacing ($dx$, $dy$ or $dz$). Although every output file
has associate with it an time $t$, this number is not output in the
file. Instead the output file number and is referred output time is
informed in the log output. 

An example of a truncated output file
can be viewed in appendix \ref{apx:output_file}.  More information can
be found the in the supplementary material``Userguide''.

\section{Test problems:}

To validate our code we performed a few standard simulations. 

\subsection{Bulk Nematic:}

One of the 

\subsection{Cholesteric Slab:}

It is know that a cholesteric liquid crystal with pitch $p_0$ placed
inside slab with planar anchoring in both substrates will organize
itself with the profile 

\subsection{Nematic sphere with strong anchoring}


\subsection{Cholesteric sphere with weak anchoring}



\section{Software implementation and extension:}

\section{Conclusions}

In conclusion, we have presented marlics. Marlics is open source and
offers some of the most common cases for device simulation. Moreover
it provides a class organized framework where the user can program its
on cases if necessary.  

\label{sec:conclusions}

%% The Appendices part is started with the command \appendix;
%% appendix sections are then done as normal sections
 \appendix

\section{List of available parameters and its units:}
 \label{apx:input_file}

 \onecolumn
 \begin{center}
	\begin{longtable}{|c|c|c|c|}
          \hline 
          Parameter name  & variable type	& units & mandatory/standard value\\ 
          \hline 
          {Geometry}	& string & 	& Yes  \\ 
          \hline 
          {Nx}	& integer &	& Yes \\ 
          \hline                                  
          {Ny}	& integer &      & Yes \\ 
          \hline                                  
          {Nz}	& integer &      & Yes \\ 
          \hline                                  
          {dx}	& real & nm      & Yes \\ 
          \hline                                  
          {dy}	& real & nm   & Yes \\ 
          \hline                                  
          {dz}	& real & nm	& Yes \\ 
          \hline 
          {integrator}	& string & & Yes\\ 
          \hline 
          facmin	& real &   & 0.4\\ 
          \hline 
          facmax	& real &   & 3\\ 
          \hline 
          prefac	& real &  & 0.8\\ 
          \hline 
          Atol	& real &  &0.001\\ 
          \hline 
          Rtol & real &  &0.001\\ 
          \hline 
          {a} & real& MJ$/(m^2 K)$ & Yes\\
          \hline 
          {b} & real& MJ$/m^2$ & Yes\\ 
          \hline
          {c} & real& MJ$/m^2$ & Yes\\ 
          \hline 
          {K1} & real &  pN  &  See \ref{sec:software_usage}\\ 
          \hline
          {K2} & real &  pN  &  See \ref{sec:software_usage}\\ 
          \hline
          {K3} & real &  pN  &  See \ref{sec:software_usage}\\ 
          \hline 
          {L1} & real &  pN & See \ref{sec:software_usage}\\ 
          \hline 
          {L2} & real &  pN & No   \\ 
          \hline
          { L3} & real & pN & No   \\ 
          \hline
          { Ls} & real & pN & No   \\ 
          \hline
          { Lq} & real & pN/m & No   \\ 
          \hline 
          {p0 or q0} & real & nm or 1/nm & No \\ 
          \hline
          {T}  & real & K	& No \\ 
          \hline 
          {Mu or gamma} & real & Pa/s & Yes \\ 
          \hline 
          {Mu\_s or gamma\_s} & real & nm Pa m/s&  See \ref{sec:software_usage}\\ 
          \hline 
          {ti}& real & $\mu$s  & 0.0 \\ 
          \hline 
          {tf}& real & $\mu$s  & Yes\\ 
          \hline 
          {dt}& real & $\mu$s  & Tf/1e6 \\ 
          \hline 
          {timeprint}& real & $\mu$s &  Tf/20 \\ 
          \hline 
          {timeprint\_type}& string &  & Linear\\ 
          \hline 
          {timeprint\_increase\_factor}& real&   & Tf/20 \\ 
          \hline 
          {output\_folder}&	string & & . \\ 
          \hline 
          {output\_fname}&	string & & director\_field\_\$\$.csv \\ 
          \hline 
          {initial\_output\_file\_number} & int & & 0 \\ 
          \hline 
          {initial\_conditions} & string & & yes \\ 
          \hline 
          {initial\_file\_name} & string & & See \ref{sec:software_usage} \\ 
          \hline 
          {theta\_i} &	real & degrees & See \ref{sec:software_usage} \\ 
          \hline 
          {phi\_i} &	real & degrees  & See \ref{sec:software_usage}\\ 
          \hline 
          {anchoring\_type} & int + string & & See \ref{sec:software_usage}\\ 
          \hline 
          {Wo1}& int + real & &  See \ref{sec:software_usage}  \\ 
          \hline 
          {theta\_0} &  int + real  & & See \ref{sec:software_usage} \\ 
          \hline 
          {phi\_0} &	 int + real  & & See \ref{sec:software_usage} \\ 
          \hline 
	\end{longtable} 
\end{center}
\twocolumn
 
 \section{Input File example:}
 \label{apx:input_file}

 Here you can find the complete input file:
\onecolumn
\begin{lstlisting}	

   #Geometry Parameters:
   geometry  slab
   Nx  200                /*      grid size      */
   Ny  200                /*      grid size      */
   Nz  100                /*      grid size      */
   dx  10.0               /*      10^-9 m         */
   dy  10.0               /*      10^-9 m         */
   dz  10.0               /*      10^-9 m         */


  #Integrator parameters:
  integrator  DP5
  atol 0.005
  rtol 0.005
  facmax 3.0
  facmin 0.4
  prefac 0.8


  #Liquid crystal parameters:
  a   0.182
  b  -2.12
  c  1.73
  T -1            Kelvin
  k11  16.7        /*   pN   */
  k22  7.8         /*   pN   */
  k33  18.1        /*   pN   */
  k24  0           /*   pN   */ 
  p0  500
  mu_1     0.3          /*     Pa s            */
  mu_1_s   30.0          /*    Pa nm s     */


  #Time parameters:
  dt  0.001            /*     10^-6 s         */	
  ti  0.0              /*     10^-6 s         */	
  tf  5000.0           /*     10^-6 s         */

  #Output Parameters:
   time_print_type             logarithmic
   timeprint                   50.        /*  10^-6 s   */
   timeprint_increase_factor   1.16            
   output_folder               .
   output_fname                output_$$.csv
   initial_output_file_number  0	

   #Initial conditions:
   initial_conditions  random


   #Boundaries conditions

   #Bottom boundaries:
   anchoring_type 0  Rapini-Papoular
   Wo1            0  1000.0
   theta_0        0  45.0
   phi_0          0  45.0	

   #Top boundaries:
   anchoring_type 1  Fournier-Galatola
   Wo1            1  1000.0

\end{lstlisting}
\twocolumn

\section{Output file:}\label{apx:output_file}
 
%% References
%%
%% Following citation commands can be used in the body text:
%% Usage of \cite is as follows:
%%   \cite{key}         ==>>  [#]
%%   \cite[chap. 2]{key} ==>> [#, chap. 2]
%%

%% References with bibTeX database:

\bibliographystyle{elsarticle-num}
\bibliography{marlics}

%% Authors are advised to submit their bibtex database files. They are
%% requested to list a bibtex style file in the manuscript if they do
%% not want to use elsarticle-num.bst.

%% References without bibTeX database:

% \begin{thebibliography}{00}

%% \bibitem must have the following form:
%%   \bibitem{key}...
%%

% \bibitem{}

% \end{thebibliography}


\end{document}

%%
%% End of file \documentclass[a4paper,onecolumn,12pt]{article}

